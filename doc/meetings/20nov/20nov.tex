\documentclass[8pt]{beamer}
\usepackage[utf8]{inputenc}
\usepackage{amsmath}
\usepackage{amsfonts}
\usepackage{xcolor,colortbl}
\usepackage{listings}
\usepackage{graphicx}
\usepackage{picture}
\usepackage{multirow}
\usepackage{hyperref}
%\usepackage{eepic}
%\usepackage{pstricks}
\usepackage{tikz}
\hypersetup{
    bookmarks=true,         % show bookmarks bar?
    unicode=false,          % non-Latin characters in Acrobat’s bookmarks
    pdftoolbar=true,        % show Acrobat’s toolbar?
    pdfmenubar=true,        % show Acrobat’s menu?
    pdffitwindow=false,     % window fit to page when opened
    pdfstartview={FitH},    % fits the width of the page to the window
    pdftitle={My title},    % title
    pdfauthor={Author},     % author
    pdfsubject={Subject},   % subject of the document
    pdfcreator={Creator},   % creator of the document
    pdfproducer={Producer}, % producer of the document
    pdfkeywords={keyword1} {key2} {key3}, % list of keywords
    pdfnewwindow=true,      % links in new window
    colorlinks=true,       % false: boxed links; true: colored links
    linkcolor=red,          % color of internal links (change box color with linkbordercolor)
    citecolor=green,        % color of links to bibliography
    filecolor=magenta,      % color of file links
    urlcolor=cyan           % color of external links
}

\title{INF3320 Group Meeting}
\author{Jens Kristoffer Reitan Markussen}
\date{\today}
\subject{Informatics}
\begin{document}
\lstset{
      language=C,
      tabsize=2, 
      morekeywords={vec2, vec3, vec4, mat3, mat4, complex}
  }  

	\begin{frame}
	\frametitle{Exercises - Programming}
	\framesubtitle{Mandelbrot set computed and visualized on GPU}
	\includegraphics[scale=0.35]{brot.png}
	
	\end{frame}
	
	\begin{frame}[fragile]
	\frametitle{Exercises - Programming}
	\framesubtitle{Mandelbrot set computed and visualized on GPU}
	Definition: $P_c:z\rightarrow z^2 + c$\\ 
	For all complex numbers $c$, for which $z$ does not tend towards infinity. \\
	- If the absolute value of $z$ is larger than $2$, we tend towards infinity.  
	\\ \bigskip
	How to compute:
	\begin{enumerate}
	\item Use pixel coordinate as complex $c$
	\item Set $z_0 = 0$
	\item Compute $z_{n+1} = {z_n}^2 + c$
	\item If $|z_n| > 2$ for any $n$, abort
	\item If $n > N$, where $N$ is the max number of iterations, abort.
	\end{enumerate}
	
	\bigskip
	GLSL does not support a complex datatype, use a 2D vector type instead. We compute the absolute value of a complex number the same way as we compute the magnitude of a 2D vector, so we can use the length() function in GLSL. \\ \bigskip
	Implement computing and visualizing of the mandelbrot set using the fragment shader, you can start from \verb|ex15/mandelbrot.frag.template|
	
	\end{frame}	
	
	\begin{frame}[fragile]
	\frametitle{Exercises - Programming}
	\framesubtitle{Mandelbrot set computed and visualized on GPU}
	\begin{lstlisting}
	//Initialize z to be c, the complex coordinate
	complex z = c;
	
	//Iterate until the length of z is larger than
	//two, or until we have reached max_iterations
	while (|z| < 2 && i < max_iterations) {
	  //Update the value of z according to the
	  //formula
	  z = z^2 + c
	  ++i;
	}
	
	if (|z| < 2) {
	  out_color = 1; //Assumed to be part of the set
	}
	else {
	  out_color = 0; //Not part of the set
	}
	\end{lstlisting}
	
	\end{frame}
	
	\begin{frame}[fragile]
	\frametitle{Exercises - Programming}
	\framesubtitle{Mandelbrot set computed and visualized on GPU}
	Black and white is dull, we can use linear interpolation to visualize the fractal with colors
	\begin{lstlisting}
	if (i < max_iterations) {
	  float t = (i-log(log(|z|)/log(2))/log(2)) / max_iterations;
	  out_color = mix(red,green,t); //Assumed to be part of the set
	}
	else {
	  out_color = 0; //Not part of the set
	}	
	\end{lstlisting}
	\end{frame}
	
	\begin{frame}[fragile]
	\frametitle{Exercises - Programming}
	\framesubtitle{Mandelbrot set computed and visualized on GPU}
	Interpolating over the RGB color space doesn't produce very interesting results, instead we can interpolate over the HSV color space, then convert it to RGB.
	\url{http://en.wikipedia.org/wiki/HSL_and_HSV}\\ \bigskip
	Try to interpolate between $[0,1,1]$ and $[2\pi,1,1]$ in HSV, then convert it to RGB
	\end{frame}

\end{document}