\documentclass[11pt]{beamer}
\usepackage[utf8]{inputenc}
\usepackage{amsmath}
\usepackage{amsfonts}
\usepackage{xcolor,colortbl}
\usepackage{listings}
\usepackage{graphicx}
\usepackage{multirow}
\usepackage{hyperref}
\hypersetup{
    bookmarks=true,         % show bookmarks bar?
    unicode=false,          % non-Latin characters in Acrobat’s bookmarks
    pdftoolbar=true,        % show Acrobat’s toolbar?
    pdfmenubar=true,        % show Acrobat’s menu?
    pdffitwindow=false,     % window fit to page when opened
    pdfstartview={FitH},    % fits the width of the page to the window
    pdftitle={My title},    % title
    pdfauthor={Author},     % author
    pdfsubject={Subject},   % subject of the document
    pdfcreator={Creator},   % creator of the document
    pdfproducer={Producer}, % producer of the document
    pdfkeywords={keyword1} {key2} {key3}, % list of keywords
    pdfnewwindow=true,      % links in new window
    colorlinks=true,       % false: boxed links; true: colored links
    linkcolor=red,          % color of internal links (change box color with linkbordercolor)
    citecolor=green,        % color of links to bibliography
    filecolor=magenta,      % color of file links
    urlcolor=cyan           % color of external links
}

\title{INF3320 Group Meeting}
\author{Jens Kristoffer Reitan Markussen}
\date{\today}
\subject{Informatics}
\begin{document}

	\begin{frame}
	\frametitle{Exercises}
	Continue working with the exercises from last weeks meeting if
	you didn't finish them, solutions for all tasks except the extra task
	are published on the git repository.
	\end{frame}

	\begin{frame}
	\frametitle{Exercises}
	\framesubtitle{From exercises2.pdf}
	\begin{enumerate}
	\item Make a program that draws a sequence of points joined together by line segments. The
user should be able to click on a point with the mouse and drag it around.
This program forms the basis for a later exercise involving curves.
In the function myMousePress you need to find out which of the points in points is
nearest the cursor, and if the point is close enough, update selected with the index of
this point.
In the function myMouseMotion we check whether selected is different from -1. If
it is then we update the point with the given index to the position of the cursor.
If the user clicks on a line segment, a new point is created between the end points of the
segment thus splitting the segment into two. If the user clicks with the right mouse button
on a point the point is removed.
You can start from the file ex2-4\_polylinemanip.cpp.template. \\
See also LineSegment.hpp

	\end{enumerate}
	\end{frame}
	
	\begin{frame}
	\frametitle{Exercises}
	\framesubtitle{From exercises4.pdf}
	\begin{enumerate}
	\item[1] Formulate a sequence of OpenGL calls that sets the MODELVIEW matrix to represent a rotation of $\theta$ around the axis given by the two points $[1,1,4]$ and $[3,4,7]$. Also formulate a sequence of GLM matrix operations that achieves the same result.
	

	\end{enumerate}
	\end{frame}
	
	\begin{frame}
	\begin{enumerate}
\item[2] Given a bounding box defined by
	\begin{align*}
	\mathbb{B}=\{[x,y,z]:-1\leq x\leq 1\, -1\leq y\leq 1\, -3\leq x\leq -1\}
	\end{align*}
The MODELVIEW and PROJECTION matrices are identity matrices.
Formulate a sequence of calls to glFrustum and other transformation functions s.t. the
complete bounding box is inside the frustum. Also formulate a sequence of GLM matrix
operations that achieves the same result.	
	
	\item[3] Then, assume the bounding box are given by
	\begin{align*}
	\mathbb{B}=\{[x,y,z]:-1\leq x\leq 1\, -1\leq y\leq 1\, -1\leq x\leq 1\}
	\end{align*}	
	 The MODELVIEW and PROJECTION matrices are identity matrices.
What is the simplest sequence of OpenGL calls s.t. the complete bounding box is inside the
frustum? What is the sequence of GLM matrix operations that achieves the same result?

	\end{enumerate}
	\end{frame}
	
	\begin{frame}
	\frametitle{Exercises}
	Placeholder for robot-arm exercise if I manage to finish preparing it in time.
	\end{frame}

\end{document}