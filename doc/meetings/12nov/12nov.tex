\documentclass[11pt]{beamer}
\usepackage[utf8]{inputenc}
\usepackage{amsmath}
\usepackage{amsfonts}
\usepackage{xcolor,colortbl}
\usepackage{listings}
\usepackage{graphicx}
\usepackage{picture}
\usepackage{multirow}
\usepackage{hyperref}
%\usepackage{eepic}
%\usepackage{pstricks}
\usepackage{tikz}
\hypersetup{
    bookmarks=true,         % show bookmarks bar?
    unicode=false,          % non-Latin characters in Acrobat’s bookmarks
    pdftoolbar=true,        % show Acrobat’s toolbar?
    pdfmenubar=true,        % show Acrobat’s menu?
    pdffitwindow=false,     % window fit to page when opened
    pdfstartview={FitH},    % fits the width of the page to the window
    pdftitle={My title},    % title
    pdfauthor={Author},     % author
    pdfsubject={Subject},   % subject of the document
    pdfcreator={Creator},   % creator of the document
    pdfproducer={Producer}, % producer of the document
    pdfkeywords={keyword1} {key2} {key3}, % list of keywords
    pdfnewwindow=true,      % links in new window
    colorlinks=true,       % false: boxed links; true: colored links
    linkcolor=red,          % color of internal links (change box color with linkbordercolor)
    citecolor=green,        % color of links to bibliography
    filecolor=magenta,      % color of file links
    urlcolor=cyan           % color of external links
}

\title{INF3320 Group Meeting}
\author{Jens Kristoffer Reitan Markussen}
\date{\today}
\subject{Informatics}
\begin{document}

	\begin{frame}
	\frametitle{Exercises - Theoretical}
	\framesubtitle{Loop Subdivision}	
	
	\setlength{\unitlength}{5cm}
\scalebox{0.68}{	
	\begin{picture}(0.5,0.5)(-0.5,-0.5)
	\thicklines
	\put(0.0,0.0){\circle*{1}}
	\put(0.08,0.01){0.0,0.0}
	\put(0.0,0.0){\line(1,1){0.3}}
	\put(0.0,0.0){\line(-1,1){0.3}}
	\put(0.0,0.0){\line(1,0){0.5}}
	\put(0.0,0.0){\line(-1,0){0.5}}
	
	\put(0.0,0.0){\line(-1,-1){0.3}}
	\put(0.0,0.0){\line(1,-1){0.3}}
	
	\put(-0.3,-0.3){\line(1,0){0.6}}
	\put(-0.3,-0.3){\circle*{1}}
	\put(-0.22,-0.28){-0.3,-0.3}
	
	\put(0.3,-0.3){\line(2,3){0.2}}
	\put(0.3,-0.3){\circle*{0.2}}
	\put(0.38,-0.28){0.3,-0.3}
	
	\put(0.5,0.0){\line(-2,3){0.2}}
	\put(0.5,0.0){\circle*{0.2}}
	\put(0.58,0.01){0.5,0.0}
	
	\put(0.3,0.3){\line(-1,0){0.6}}
	\put(0.3,0.3){\circle*{1}}
	\put(0.38,0.32){0.3,-0.3}
	
	\put(-0.3,0.3){\line(-2,-3){0.2}}
	\put(-0.3,0.3){\circle*{0.2}}
	\put(-0.22,0.22){-0.3,0.3}
	
	\put(-0.5,0.0){\line(2,-3){0.2}}
	\put(-0.5,0.0){\circle*{0.2}}
	\put(-0.45,0.01){-0.5,0.0}
	
	\end{picture}
	}
	\\Use pen and paper and apply the loop subdivision scheme to this triangle mesh
	\begin{enumerate}
	\item How many new triangles do we get for each old triangle?
	\item Write pseudo (or C/C++) code for the half-edge adjacency queries needed to create the new vertices/triangles
	\item How do we handle the boundary?
	\end{enumerate}
	\end{frame}

	\begin{frame}
	\frametitle{Exercises - Theoretical}
	\framesubtitle{$\sqrt{3}$ Subdivision}
	
	\setlength{\unitlength}{5cm}
\scalebox{0.68}{	
	\begin{picture}(0.5,0.5)(-0.5,-0.5)
	\thicklines
	\put(0.0,0.0){\circle*{1}}
	\put(0.08,0.01){0.0,0.0}
	\put(0.0,0.0){\line(1,1){0.3}}
	\put(0.0,0.0){\line(-1,1){0.3}}
	\put(0.0,0.0){\line(1,0){0.5}}
	\put(0.0,0.0){\line(-1,0){0.5}}
	
	\put(0.0,0.0){\line(-1,-1){0.3}}
	\put(0.0,0.0){\line(1,-1){0.3}}
	
	\put(-0.3,-0.3){\line(1,0){0.6}}
	\put(-0.3,-0.3){\circle*{1}}
	\put(-0.22,-0.28){-0.3,-0.3}
	
	\put(0.3,-0.3){\line(2,3){0.2}}
	\put(0.3,-0.3){\circle*{0.2}}
	\put(0.38,-0.28){0.3,-0.3}
	
	\put(0.5,0.0){\line(-2,3){0.2}}
	\put(0.5,0.0){\circle*{0.2}}
	\put(0.58,0.01){0.5,0.0}
	
	\put(0.3,0.3){\line(-1,0){0.6}}
	\put(0.3,0.3){\circle*{1}}
	\put(0.38,0.32){0.3,-0.3}
	
	\put(-0.3,0.3){\line(-2,-3){0.2}}
	\put(-0.3,0.3){\circle*{0.2}}
	\put(-0.22,0.22){-0.3,0.3}
	
	\put(-0.5,0.0){\line(2,-3){0.2}}
	\put(-0.5,0.0){\circle*{0.2}}
	\put(-0.45,0.01){-0.5,0.0}
	
	\end{picture}
	}
	\\Use pen and paper and apply the $\sqrt{3}$ subdivision scheme to this triangle mesh
	\begin{enumerate}
	\item How many new triangles do we get for each old triangle?
	\item Write pseudo (or C/C++) code for the half-edge adjacency queries needed to create the new vertices/triangles
	\item How do we handle the boundary?
	\end{enumerate}	
	\end{frame}

\end{document}