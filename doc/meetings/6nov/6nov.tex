\documentclass[11pt]{beamer}
\usepackage[utf8]{inputenc}
\usepackage{amsmath}
\usepackage{amsfonts}
\usepackage{xcolor,colortbl}
\usepackage{listings}
\usepackage{graphicx}
\usepackage{multirow}
\usepackage{hyperref}
\hypersetup{
    bookmarks=true,         % show bookmarks bar?
    unicode=false,          % non-Latin characters in Acrobat’s bookmarks
    pdftoolbar=true,        % show Acrobat’s toolbar?
    pdfmenubar=true,        % show Acrobat’s menu?
    pdffitwindow=false,     % window fit to page when opened
    pdfstartview={FitH},    % fits the width of the page to the window
    pdftitle={My title},    % title
    pdfauthor={Author},     % author
    pdfsubject={Subject},   % subject of the document
    pdfcreator={Creator},   % creator of the document
    pdfproducer={Producer}, % producer of the document
    pdfkeywords={keyword1} {key2} {key3}, % list of keywords
    pdfnewwindow=true,      % links in new window
    colorlinks=true,       % false: boxed links; true: colored links
    linkcolor=red,          % color of internal links (change box color with linkbordercolor)
    citecolor=green,        % color of links to bibliography
    filecolor=magenta,      % color of file links
    urlcolor=cyan           % color of external links
}

\title{INF3320 Group Meeting}
\author{Jens Kristoffer Reitan Markussen}
\date{\today}
\subject{Informatics}
\begin{document}

	\begin{frame}
	\frametitle{Exercises - Theoretical}
	\framesubtitle{Half-edge data structure}
	\begin{enumerate}
	\item Sketch up the half-edge data structure in pseudo (or C/C++) code 
	\item Some adjacency queries we might want to make on a triangle mesh are:
		\begin{enumerate}
		\item[a] Which triangles use this vertex?
		\item[b]	 Which triangles border this edge?
		\item[c]	 Which triangles are adjacent to this triangle?
		\item[d] Which vertices are adjacent to this vertex?
		\end{enumerate}	 
		Write pseudo (or C/C++) code for how you would iterate over the the half-edge data structure to find the results of these queries.	
		
	\end{enumerate}
	\end{frame}

	\begin{frame}
	\frametitle{Exercises - Theoretical}
	\framesubtitle{From exercises10.pdf}
	\begin{enumerate}
	\item  Let $p_0=(-1,1)$, $p_1=(1,1)$, $p_2=(1,0)$ be the control points of a quadratic Bezier curve $p$. Evaluate $p$ at $t=\frac{1}{4}$ using the de Casteljau algorithm

	\item Express a quadratic Bezier curve $p(t)=\sum^2_{i=0}p_iB_{0,2}(t)$ in monomial form, i.e., in the form $p(t)=a_0+a_1t+a_2t^2$
	
	\item Express a quadratic polynomial $p(t)=a_0+a_1t+a_2t^2$ in Bezier form, i.e., in the form $p(t)=\sum^2_{i=0}p_iB_{0,2}(t)$
	\end{enumerate}
	\end{frame}
	
	\begin{frame}
	\frametitle{Exercises - Programming}
	\framesubtitle{From exercises10.pdf}
	\begin{center}
	 Start from
ex7-6\_bezier.cpp.template
and implement the function
deCasteljauEval
which applies the de Casteljau algorithm to the Bezier curve defined
by
src\_points
at the parameter value
t
(the degree of the curve is implicitly given by
how many points there are).
	\end{center}
	\end{frame}

\end{document}